\documentclass{scrartcl}
\usepackage{fontspec} %connects to native fonts
\usepackage{amsmath}
\usepackage{mathtools}
\usepackage{cleveref}
\usepackage{pgfplots}
\usepackage{graphicx}
\usepackage{wrapfig}
\usepackage{fancyref}
\usepackage{amssymb}
\usepackage{subfig}
\usepackage{float}
\usepackage[justification=RaggedRight, singlelinecheck=false, font={footnotesize}]{caption}
\usepackage[portuguese]{babel}
\usepackage[title,titletoc,toc]{appendix}


\usepackage{lipsum}
\usepackage{blindtext}
\addtokomafont{sectioning}{\rmfamily}

\begin{document}
\pagenumbering{arabic}
\bibliographystyle{plain}
\title{
	\textnormal{
	\LARGE Universidade de Lisboa - Instituto Superior Técnico\\
	\Large Licenciatura em Engenharia Informática e de Computadores\\
	\Large Inteligência Artificial
\\}
	\LARGE1º Projeto
	\vspace{-1ex}
	}
\author{Gonçalo Marques,
	\texttt{84719}
	\and
	Manuel Sousa,
	\texttt{84740}
}
\date{	\vspace{-1ex}
		\vspace{-4ex}
	}
\maketitle

\section*{Introdução}
balbalbalablablball

\section*{Procura em Profundidade Primeiro}
Nao Completa
Nao Optima
O(b^m) tempo \par
O(b*m) espaco

\section*{Procura Gananciosa}
Nao completa (pode entrar em ciclo)\par
Tempo(b^m) mas uma boa heurística pode reduzi-lo dramaticamente
Espaco(b^m) no pior caso mantém todos os nós em memória
Nao Optima\par


fdgfgggdfdfggfddgfgdffdggdf

\section*{A*}
Optima!
Completa
Tempo: Exponencial 
Exponencial: mantém todos os nós em memória (no pior caso)

sdfsfsdsfdsdfsdfsdf

\section*{Conclusão}
adsdadsdasdsadasdasdas

sdfsfsdsfdsdfsdfsdf

\end{document}
