\documentclass{scrartcl}
\usepackage{fontspec} %connects to native fonts
\usepackage{amsmath}
\usepackage{mathtools}
\usepackage{cleveref}
\usepackage{pgfplots}
\usepackage{graphicx}
\usepackage{wrapfig}
\usepackage{fancyref}
\usepackage{amssymb}
\usepackage{subfig}
\usepackage{float}
\usepackage[justification=RaggedRight, singlelinecheck=false, font={footnotesize}]{caption}
\usepackage[portuguese]{babel}
\usepackage[title,titletoc,toc]{appendix}


\usepackage{lipsum}
\usepackage{blindtext}
\addtokomafont{sectioning}{\rmfamily}

\begin{document}
\pagenumbering{arabic}
\bibliographystyle{plain}
\title{
	\textnormal{
	\LARGE Universidade de Lisboa - Instituto Superior Técnico\\
	\Large Licenciatura em Engenharia Informática e de Computadores\\
	\Large Inteligência Artificial
\\}
	\LARGE1º Projeto
	\vspace{-1ex}
	}
\author{Gonçalo Marques,
	\texttt{84719}
	\and
	Manuel Sousa,
	\texttt{84740}
}
\date{	\vspace{-1ex}
		\vspace{-4ex}
	}
\maketitle

\section*{Introdução}
balbalbalablablball

\section*{Descrição do problema}
sdfsdfsdfsdffdssffsdfdssdfsdfdsfsfsdfdsfsfs\par

\section*{Procura em Profundidade Primeiro}

	\begin{table}[h!]
	  \centering
	  \caption{Desempenho}
	  \label{tab:Procura em Profundidade Primeiro}
	  \begin{tabular}{l|c|c|c|c|r}
	     & Exemplo 1 & Exemplo 2 & Exemplo 3 & Exemplo 4 & Exemplo 5 \\
	    \hline
	    Tempo de Execução (s) & 0.000291 & 0.001387 & 18.795010 & 0.027842 & 795.923310 \\
	    \hline
	    Nº de nós expandidos & 0 & 7 & 74701 & 85 & 3123363 \\
	    \hline
	    Nº de nós gerados & 1 & 4 & 74702 & 54 & 3123308 \\
	    \hline
	  \end{tabular}
	\end{table}

\section*{Procura Gananciosa}

	\begin{table}[h!]
	  \centering
	  \caption{Desempenho}
	  \label{tab:Procura Gananciosa}
	  \begin{tabular}{l|c|c|c|c|r}
	     & Exemplo 1 & Exemplo 2 & Exemplo 3 & Exemplo 4 & Exemplo 5 \\
	    \hline
	    Tempo de Execução (s) & 0.000331 & 0.001410 & 26.159087 & 0.016135 & 0.119890 \\
	    \hline
	    Nº de nós expandidos & 0 & 6 & 74701 & 59 & 319 \\
	    \hline
	    Nº de nós gerados & 1 & 3 & 74702 & 42 & 256 \\
	    \hline
	  \end{tabular}
	\end{table}

\section*{A*}

	\begin{table}[h!]
	  \centering
	  \caption{Desempenho}
	  \label{tab:Procura Gananciosa}
	  \begin{tabular}{l|c|c|c|c|r}
	     & Exemplo 1 & Exemplo 2 & Exemplo 3 & Exemplo 4 & Exemplo 5 \\
	    \hline
	    Tempo de Execução (s) & 0.000266 & 0.001522 & 25.653358 & 0.015687 & 0.040954 \\
	    \hline
	    Nº de nós expandidos & 0 & 7 & 74701 & 43 & 91 \\
	    \hline
	    Nº de nós gerados & 1 & 4 & 74702 & 24 & 16 \\
	    \hline
	  \end{tabular}
	\end{table}

\section*{Conclusão}



\end{document}
