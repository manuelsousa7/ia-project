\documentclass{scrartcl}
\usepackage{fontspec} %connects to native fonts
\usepackage{amsmath}
\usepackage{mathtools}
\usepackage{cleveref}
\usepackage{pgfplots}
\usepackage{graphicx}
\usepackage{wrapfig}
\usepackage{fancyref}
\usepackage{amssymb}
\usepackage{subfig}
\usepackage{float}
\usepackage[justification=RaggedRight, singlelinecheck=false, font={footnotesize}]{caption}
\usepackage[portuguese]{babel}
\usepackage[title,titletoc,toc]{appendix}


\usepackage{lipsum}
\usepackage{blindtext}
\addtokomafont{sectioning}{\rmfamily}

\begin{document}
\pagenumbering{arabic}
\bibliographystyle{plain}
\title{
	\textnormal{
	\LARGE Universidade de Lisboa - Instituto Superior Técnico\\
	\Large Licenciatura em Engenharia Informática e de Computadores\\
	\Large Inteligência Artificial
\\}
	\LARGE1º Projeto
	\vspace{-1ex}
	}
\author{Gonçalo Marques,
	\texttt{84719}
	\and
	Manuel Sousa,
	\texttt{84740}
}
\date{	\vspace{-1ex}
		\vspace{-4ex}
	}
\maketitle

\section*{Procura em Profundidade Primeiro}
A Procura em Profundidade Primeiro é um algoritmo de procura não optimo. Este algoritmo não encontra soluções em espaços de estados com profundidade infinita ou com ciclos, e por isso concluimos que não é completo.\par
Comparando o Exemplo 2 e o Exemplo 4, conclui-se que o tamanho do tabuleiro tem um impacto significativo na procura, aumentando o tempo de execução para 2000\% no Exemplo 4.
Através de outra comparação entre o Exemplo 4 e o Exemplo 5, concluímos que o aumento do
número de cores tem um impacto extremamente significativo no desempenho, aumentando o tempo
de execução para cerca de 3000000\%.\par
	\begin{table}[h!]
	  \centering
	  \caption{Desempenho da Procura em Profundidade Primeiro}
	  \label{tab:Profundidade Primeiro}
	  \begin{tabular}{l|c|c|c|c|r}
	     & Exemplo 1 & Exemplo 2 & Exemplo 3 & Exemplo 4 & Exemplo 5 \\
	    \hline
	    Tempo de Execução (s) & 0.000291 & 0.001387 & 18.795010 & 0.027842 & 795.923310 \\
	    \hline
	    Nº de nós expandidos & 1 & 4 & 74702 & 54 & 3123308 \\
	    \hline
	    Nº de nós gerados & 0 & 7 & 74701 & 85 & 3123363 \\
	    \hline
	  \end{tabular}
	\end{table}
	\par

\section*{Procura Gananciosa}
A Procura gananciosa é um algoritmo batante semelhante à procura em profundidade primeiro, no entanto mais exigente em memória.Não é optima, e tambem não é completa, visto que pode entrar em ciclo. A procura gananciosa vai procurar sempre o nó que julga estar mais perto do objetivo final.\par
Relativamente ao tamanho do tabuleiro, o impacto é semelhante ao algoritmo anterior, apesar de ser ligeiramente mais significativo.
Em relação ao impacto no aumento do número de cores, conclui-se que este é significativo,
aumentando o tempo de execução para cerca de 150\%.\par


	\begin{table}[h!]
	  \centering
	  \caption{Desempenho da Procura Gananciosa}
	  \label{tab:Procura Gananciosa}
	  \begin{tabular}{l|c|c|c|c|r}
	     & Exemplo 1 & Exemplo 2 & Exemplo 3 & Exemplo 4 & Exemplo 5 \\
	    \hline
	    Tempo de Execução (s) & 0.000331 & 0.001410 & 26.159087 & 0.016135 & 0.119890 \\
	    \hline
	    Nº de nós expandidos & 1 & 3 & 74702 & 42 & 256 \\
	    \hline
	    Nº de nós gerados & 0 & 6 & 74701 & 59 & 319 \\
	    \hline
	  \end{tabular}
	\end{table}
	\par

\section*{A*}

A procura por $A*$ é um algoritmo de procura que oferece optimalizade. No entanto isso apenas acontece se garantirmos que a heuristica utilizada é admissivel. Um dos principais problemas da utilização desta procura é que é necessário guardar todos os nós em memória, tornando a complexidade espacial exponencial.\par

Comparando o Exemplo 2 e o Exemplo 4, conclui-se que o tamanho do tabuleiro tem um impacto bastante significativo na procura, visto que duplicando o tamanho do tabuleiro, o
tempo de procura aumentou para 1000\%.
Através de outra comparação entre o Exemplo 4 e o Exemplo 5, concluímos que o aumento do
número de cores tem um impacto no tempo de procura, mas não tão significativo.
Quase duplicando o número de cores, obteve-se um aumento do tempo de procura de cerca de 250\%.\par

	\begin{table}[h!]
	  \centering
	  \caption{Desempenho do A*}
	  \label{tab:A*}
	  \begin{tabular}{l|c|c|c|c|r}
	     & Exemplo 1 & Exemplo 2 & Exemplo 3 & Exemplo 4 & Exemplo 5 \\
	    \hline
	    Tempo de Execução (s) & 0.000266 & 0.001522 & 25.653358 & 0.015687 & 0.040954 \\
	    \hline
	    Nº de nós expandidos & 1 & 4 & 74702 & 24 & 16 \\
	    \hline
	    Nº de nós gerados & 0 & 7 & 74701 & 43 & 91 \\
	    \hline
	  \end{tabular}
	\end{table}
	\par

\section*{Conclusão}

Sendo $b$ o fator de ramificação, e $m$ a profundidade máxima, podemos concluir resumidamente numa tabela, as várias propriedades dos algoritmos aqui testados:

	\begin{table}[h!]
	  \centering
	  \caption{Propriedades das várias procuras}
	  \label{tab: Comparação}
	  \begin{tabular}{l|c|c|c|r}
	     & C. Temporal & C. Espacial & Completo & Ótimo \\
	    \hline
	    Profundidade Primeiro & $O(b^m)$ & $O(b*m)$ & Não & Não \\
	    \hline
	    Procura Gananciosa & $O(b^m)$ & $O(b^m)$ & Não & Não \\
	    \hline
	    A* & Exponencial & Exponencial & Sim(Exceto se for infinito) & Sim \\
	    \hline
	  \end{tabular}
	\end{table}

A heurística usada neste projeto (nº de grupos restantes) não é admissível, porque em certos casos vai sobrestimar o custo de atingir o objetivo. Como o problema é NP-Completo, podemos garantir pela conjetura de $P \neq NP$, que não existe uma heuristica admissivel para este problema.\par

\end{document}
