\documentclass{scrartcl}
\usepackage{fontspec} %connects to native fonts
\usepackage{amsmath}
\usepackage{mathtools}
\usepackage{cleveref}
\usepackage{pgfplots}
\usepackage{graphicx}
\usepackage{wrapfig}
\usepackage{fancyref}
\usepackage{amssymb}
\usepackage{subfig}
\usepackage{float}
\usepackage[justification=RaggedRight, singlelinecheck=false, font={footnotesize}]{caption}
\usepackage[portuguese]{babel}
\usepackage[title,titletoc,toc]{appendix}


\usepackage{lipsum}
\usepackage{blindtext}
\addtokomafont{sectioning}{\rmfamily}

\begin{document}
\pagenumbering{arabic}
\bibliographystyle{plain}
\title{
	\textnormal{
	\LARGE Universidade de Lisboa - Instituto Superior Técnico\\
	\Large Licenciatura em Engenharia Informática e de Computadores\\
	\Large Inteligência Artificial
\\}
	\LARGE1º Projeto - Grupo 22
	\vspace{-1ex}
	}
\author{Gonçalo Marques,
	\texttt{84719}
	\and
	Manuel Sousa,
	\texttt{84740}
}
\date{	\vspace{-1ex}
		\vspace{-4ex}
	}
\maketitle

\section*{P1}

Texto.\par

\section*{P2}

Texto.\par

\section*{P3}

A imagem seguinte ilustra como o agente se movimenta pelo ambiente 1, incluíndo uma representação gráfica do mesmo:
	*inserir imagem Trajetória 1*

A imagem seguinte ilustra como o agente se movimenta pelo ambiente 1, incluíndo uma representação gráfica do mesmo:
	*inserir imagem Trajetória 2*

A função de recompensa é a seguinte para ambas as trajetórias *inserir imagem Piecewise 1*.\par

\end{document}
