\documentclass{scrartcl}
\usepackage{fontspec} %connects to native fonts
\usepackage{amsmath}
\usepackage{mathtools}
\usepackage{cleveref}
\usepackage{pgfplots}
\usepackage{graphicx}
\usepackage{wrapfig}
\usepackage{fancyref}
\usepackage{amssymb}
\usepackage{subfig}
\usepackage{float}
\usepackage[justification=RaggedRight, singlelinecheck=false, font={footnotesize}]{caption}
\usepackage[portuguese]{babel}
\usepackage[title,titletoc,toc]{appendix}


\usepackage{lipsum}
\usepackage{blindtext}
\addtokomafont{sectioning}{\rmfamily}

\begin{document}
\pagenumbering{arabic}
\bibliographystyle{plain}
\title{
	\textnormal{
	\LARGE Universidade de Lisboa - Instituto Superior Técnico\\
	\Large Licenciatura em Engenharia Informática e de Computadores\\
	\Large Inteligência Artificial
\\}
	\LARGE1º Projeto - Grupo 22
	\vspace{-1ex}
	}
\author{Gonçalo Marques,
	\texttt{84719}
	\and
	Manuel Sousa,
	\texttt{84740}
}
\date{	\vspace{-1ex}
		\vspace{-4ex}
	}
\maketitle

\section*{P1}

Texto.\par

\section*{P2}

Texto.\par

\section*{P3}

Trajetória do ambiente 1:\par

	\begin{table}[h!]
	  \centering
	  \caption{Trajetória 1}
	  \label{tab:Trajetória 1}
	  \begin{tabular}{|l|c|c|c|c|r|}
	  	\hline
	    5 & 0 & 6 & 0\\
	    \hline
	    6 & 0 & 6 & 1\\
	    \hline
	    6 & 0 & 6 & 1\\
	    \hline
	    6 & 0 & 6 & 1\\
	    \hline
	  \end{tabular}
	\end{table}
	\par

Trajetória do ambiente 2:\par

	\begin{table}[h!]
	  \centering
	  \caption{Trajetória 2}
	  \label{tab:Trajetória 2}
	  \begin{tabular}{|l|c|c|c|c|r|}
	  	\hline
	    5 & 0 & 6 & 0\\
	    \hline
	    6 & 0 & 1 & 1\\
	    \hline
	    1 & 0 & 2 & 0\\
	    \hline
	    2 & 1 & 1 & 0\\
	    \hline
	  \end{tabular}
	\end{table}
	\par

A função de recompensa é a função de Q-Learning (inserir função Q-Learning), com uma learning rate (alpha), ou a importância de nova informação face
à informação já aprendida, de 0.14, e um discount factor (gamma), ou a importância de recompensas atuais face a futuras, de 0.9.\par

O agente move-se sempre para o estado que tem uma maior recompensa, de acordo com a implementação da função Q2pol.\par

\end{document}
